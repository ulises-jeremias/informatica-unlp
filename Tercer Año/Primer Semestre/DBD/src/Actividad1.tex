\documentclass[a4paper,10pt]{article}


\RequirePackage{color,graphicx}
\usepackage[usenames,dvipsnames]{xcolor}
\usepackage[big]{layaureo} 				%better formatting of the A4 page
% an alternative to Layaureo can be ** \usepackage{fullpage} **
\usepackage{supertabular} 				%for Grades
\usepackage{titlesec}					%custom \section
%Setup hyperref package, and colours for links
\usepackage{hyperref}
\definecolor{linkcolour}{rgb}{0,0.2,0.6}
\hypersetup{colorlinks,breaklinks,urlcolor=linkcolour, linkcolor=linkcolour}
\usepackage[utf8]{inputenc}

%Sections inspired by:
%http://stefano.italians.nl/archives/26
\titleformat{\section}{\Large\scshape\raggedright}{}{0em}{}[\titlerule]
\titlespacing{\section}{0pt}{3pt}{3pt}
%Tweak a bit the top margin
%\addtolength{\voffset}{-1.3cm}

%Italian hyphenation for the word: ''corporations''
\hyphenation{im-pre-se}

%-------------WATERMARK TEST---------------
\usepackage[absolute]{textpos}

\setlength{\TPHorizModule}{30mm}
\setlength{\TPVertModule}{\TPHorizModule}
\textblockorigin{2mm}{0.65\paperheight}
\setlength{\parindent}{0pt}

\usepackage{vmargin}

\setpapersize{A4}
\setmargins{3.5cm}       % margen izquierdo
{2.0cm}                        % margen superior
{14.5cm}                      % anchura del texto
{24.0cm}                    % altura del texto
{10pt}                           % altura de los encabezados
{1cm}                           % espacio entre el texto y los encabezados
{0pt}                             % altura del pie de página
{2cm}                           % espacio entre el texto y el pie de página

%%%%%%%%%%%%%%%%%%%%%%%%%%%%%%%%%%%%%%%%%%%%%%%%%%%%%%%%%%%%%%%%%%


\title{Diseño de Base de Datos - Redictado 2017}
\author{Cornejo Fandos - Di Cunzolo}
\date{Marzo 2017}

\begin{document}

\maketitle

\section{Actividad 1}
    \subsection{Especificación de requerimientos}
    
    La organización COD Project cuenta con un proyecto que tiene como objetivo el desarrollo de una red social orientada a programadores, donde estos puedan compartir sus proyectos, recientes actividades y poder ver las actividades de su entorno. \\
    
    Los usuarios pueden ser, usuarios simples y organizaciones.
    
    De cada usuario simple se conocen datos personales, contraseña, email, nombre de usuario, foto de perfil, breve biografía, empresa (en caso de pertenecer a alguna), fecha de ingreso al sistema, nacionalidad, localidad. \\
    
   
    
    Todos los usuarios contarán con un muro donde mostrarán sus proyectos y ultimas actividades. \\
    De cada proyecto se conoce usuario creador, fecha de comienzo, fecha de finalización (en caso de estar finalizado), última versión estable (en caso contar con alguna), enlace a repositorio y documentación (en caso de contar con alguno), usuarios contribuidores, tags, fecha de última modificación, breve descripción, lenguajes utilizados. \\
    
    De cada publicación se conoce usuario creador, fecha de publicación, usuario que la publicó, fecha de última modificación y tags. \\
    
    Un usuario podrá modificar sus datos en cualquier momento, como también, eliminar su cuenta. Además, podrá administrar sus proyectos libremente ya sea editando información de los mismos, como también agregar y/o eliminar. \\
    
    Entre usuarios se puede interactuar mediante el envío de mensajes privados. \\
    De cada mensaje se conoce destinatarios, fecha, asunto, cuerpo, adjuntos (imágenes y documentos). \\
    

    
    

    
    
    
    
    
    de cada proyecto se conocen lenguajes, entonces se listan lenguajes que trabajaron los usuarios armando estadisticas con porcentaje de repositorios en cada lenguaje y usuarios que programaron etc etc etc


\end{document}