\documentclass[a4paper,10pt]{article}


\RequirePackage{color,graphicx}
\usepackage[usenames,dvipsnames]{xcolor}
\usepackage[big]{layaureo} 				%better formatting of the A4 page
% an alternative to Layaureo can be ** \usepackage{fullpage} **
\usepackage{supertabular} 				%for Grades
\usepackage{titlesec}					%custom \section
%Setup hyperref package, and colours for links
\usepackage{hyperref}
\definecolor{linkcolour}{rgb}{0,0.2,0.6}
\hypersetup{colorlinks,breaklinks,urlcolor=linkcolour, linkcolor=linkcolour}
\usepackage[utf8]{inputenc}

%Sections inspired by:
%http://stefano.italians.nl/archives/26
\titleformat{\section}{\Large\scshape\raggedright}{}{0em}{}[\titlerule]
\titlespacing{\section}{0pt}{3pt}{3pt}
%Tweak a bit the top margin
%\addtolength{\voffset}{-1.3cm}

%Italian hyphenation for the word: ''corporations''
\hyphenation{im-pre-se}

%-------------WATERMARK TEST---------------
\usepackage[absolute]{textpos}

\setlength{\TPHorizModule}{30mm}
\setlength{\TPVertModule}{\TPHorizModule}
\textblockorigin{2mm}{0.65\paperheight}
\setlength{\parindent}{0pt}

\usepackage{vmargin}

\setpapersize{A4}
\setmargins{3.5cm}       % margen izquierdo
{2.0cm}                        % margen superior
{14.5cm}                      % anchura del texto
{24.0cm}                    % altura del texto
{10pt}                           % altura de los encabezados
{1cm}                           % espacio entre el texto y los encabezados
{0pt}                             % altura del pie de página
{2cm}                           % espacio entre el texto y el pie de página

%%%%%%%%%%%%%%%%%%%%%%%%%%%%%%%%%%%%%%%%%%%%%%%%%%%%%%%%%%%%%%%%%%

\title{Orientación a Objetos II}
\author{Ulises J. Cornejo Fandos}
\date{Abril 2017}

\begin{document}

\maketitle

\section{Práctica 1}
\subsection{Resolución}

    \begin{enumerate}
        \item \textbf{Evaluación del protocolo de una clase}
        
            \begin{tabular}{r|l}
                \textbf{Opción 1} & Permite la creación de un rectángulo, sin embargo no es un protocolo \\
                & aceptable dado que permite de igual forma crear un polígono cualquiera \\
                & de 4 vértices, es decir que, el polígono formado por la clase no siempre \\
                & es un rectángulo por lo que la solución que plantea este protocolo no \\
                & es aceptable. \\
                & \\
            
                \textbf{Opción 2} & Es un protocolo más correcto que el anterior, en este se define un \\ 
                & rectángulo a partir de su vértice en la esquina superior izquierda, \\
                & su ancho y su alto; es decir que, simplemente cambiando la posición del \\
                & vértice podríamos mover el rectángulo a cualquier posición del plano \\ 
                & coordenado manteniendo así sus características esenciales (base, altura), \\
                & y de igual forma, re-dimensionar su ancho y alto sin afectar su condición \\
                & de rectángulo.  \\
            \end{tabular}
            
            Por estas razones es que elijo la \textbf{opción 2)} como el protocolo más indicado al momento de definir la clase Rectángulo.
            
        \item \textbf{Delegación}
        
            \begin{tabular}{r|l}
                \textbf{Opción 1} &  Si bien permite realizar lo solicitado, la opción no es correcta en \\ 
                & cuanto a delegación de tareas ya que no le delego su tarea a la secretaria \\ 
                & sino que accedo al fichero con el que ella cuenta para que lo haga el jefe. \\
                & \\
                
                \textbf{Opción 2} & Es correcta en términos de delegación porque el jefe delega \\ 
                & correctamente la tarea a la secretaria. \\
            \end{tabular}
            
            Por estas razones es que elijo la \textbf{opción 2)} como el protocolo más indicado para la resolución de este problema.
            
        \item \textbf{Polimorfismo}
        
            \begin{tabular}{r|l}
                \textbf{Opción 4} &  En términos de Polimorfismo, esta opción es ciertamente la más \\
                & correcta de todas dado que, es la única opción en la cual se delega \\
                & únicamente las tareas no comunes de las subclases y se realizan las \\ 
                & tareas comunes de ambas en la superclase. \\
            \end{tabular}
            
        \item \textbf{Manejo de Colecciones}
        
            \begin{tabular}{r|l}
                \textbf{Opción 1} &  \\
                 & \\
                
                \textbf{Opción 2} &  \\
                 & \\
                 
                \textbf{Opción 3} &  \\
                 & \\
                 
                \textbf{Opción 4} &  \\
            \end{tabular}
    \end{enumerate}
\end{document}