\documentclass[a4paper,10pt]{article}


\RequirePackage{color,graphicx}
\usepackage[usenames,dvipsnames]{xcolor}
\usepackage[big]{layaureo} 				%better formatting of the A4 page
% an alternative to Layaureo can be ** \usepackage{fullpage} **
\usepackage{supertabular} 				%for Grades
\usepackage{titlesec}					%custom \section
%Setup hyperref package, and colours for links
\usepackage{hyperref}
\definecolor{linkcolour}{rgb}{0,0.2,0.6}
\hypersetup{colorlinks,breaklinks,urlcolor=linkcolour, linkcolor=linkcolour}
\usepackage[utf8]{inputenc}

%Sections inspired by:
%http://stefano.italians.nl/archives/26
\titleformat{\section}{\Large\scshape\raggedright}{}{0em}{}[\titlerule]
\titlespacing{\section}{0pt}{3pt}{3pt}
%Tweak a bit the top margin
%\addtolength{\voffset}{-1.3cm}

%Italian hyphenation for the word: ''corporations''
\hyphenation{im-pre-se}

%-------------WATERMARK TEST---------------
\usepackage[absolute]{textpos}

\setlength{\TPHorizModule}{30mm}
\setlength{\TPVertModule}{\TPHorizModule}
\textblockorigin{2mm}{0.65\paperheight}
\setlength{\parindent}{0pt}

\usepackage{vmargin}

\setpapersize{A4}
\setmargins{3.5cm}       % margen izquierdo
{2.0cm}                        % margen superior
{14.5cm}                      % anchura del texto
{24.0cm}                    % altura del texto
{10pt}                           % altura de los encabezados
{1cm}                           % espacio entre el texto y los encabezados
{0pt}                             % altura del pie de página
{2cm}                           % espacio entre el texto y el pie de página

%%%%%%%%%%%%%%%%%%%%%%%%%%%%%%%%%%%%%%%%%%%%%%%%%%%%%%%%%%%%%%%%%%

\title{Redes y Comunicaciones}
\author{Ulises J. Cornejo Fandos}
\date{Abril 2017}

\begin{document}

\maketitle

\section{Practica 4}
\subsection{Capa de Aplicación - Correo electrónico}

\begin{enumerate}
    \item \textbf{¿Qué protocolos se utilizan para el envío de mails entre el cliente y su servidor de correo? ¿Y entre servidores del correo?}
   
    Para el envió de mails entre el cliente y su servidor se utiliza el protocolo SMTP \textbf{\textit{(Simple Mail Transfer Protocol)}}. \\
   
   
   \item \textbf{¿Qué protocolos se utilizan para la recepción de mails? ¿Incluiría a HTTP en dichos protocolos? Enumere y explique características y diferencias entre las alternativas posibles.}
   
   Para la recepción de mails existen dos alternativas posibles: \\ \\
    \begin{tabular}{r|l}
        IMAP & Trabaja en modo de conexión permanente, es posible especificar carpetas del \\
             & lado del sevidor, visualizar mensajes de forma remota. Transmite solo la \\ 
             & cabecera del mensaje, dándole la posibilidad al usuario de borrarlo \\ 
             & directamente. El almacenamiento local del mensaje es opcional.\\ \\
        POP & Se conecta periodicamente al servidor para buscar nuevo correo, en cada \\
            & conexión se bajan todos los correos nuevos sin importar si se vayan a leer \\ 
            & o no. Por defecto elimina los mensajes del servidor, no permitiendo el \\
            & acceso a ellos desde otros dispositivos.
    \end{tabular}
   
   \item \textbf{Utilizando la VM y teniendo en cuenta los siguientes datos, abra el cliente de correo (Icedove) y configure, primero una cuenta de correo POP y luego una cuenta de correo IMAP (al crearlas, ignorar advertencias por uso de conexión sin cifrado y seleccionar Manual config).}
   
   \item \textbf{En base a lo observado. ¿Qué protocolo le parece mejor? ¿POP o IMAP? ¿Por qué? ¿Qué protocolo considera que utiliza más recursos del servidor? ¿Por qué?}
   
   \item \textbf{Relacione DNS con SMTP. Describa el proceso completo para el envío de un correo desde pepe@yahoo.com a jose@hotmail.com.}
   
   
\end{enumerate}



\end{document}