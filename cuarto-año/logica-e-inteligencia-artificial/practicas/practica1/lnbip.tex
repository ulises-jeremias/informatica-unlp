% This is lnbip.tex the demonstration file of the LaTeX macro package for
% Lecture Notes in Business Information Processing from Springer-Verlag.
% It serves as a template for authors as well.
% version 1.0 for LaTeX2e
%
\documentclass[lnbip]{svmultln}

\usepackage[utf8]{inputenc}
\usepackage{multicol}
%
\usepackage{makeidx}  % allows for indexgeneration
% \makeindex          % be prepared for an author index
%
\begin{document}
%
\mainmatter              % start of the contribution
%
\title{Lógica e Inteligencia Artificial}
%
\titlerunning{Lógica e Inteligencia Artificial - Primer Entrega}  % abbreviated title (for running head)
%                                     also used for the TOC unless
%                                     \toctitle is used
%
\author{Ulises J. Cornejo Fandos\inst{1}, Lucas Di Cunzolo\inst{2}, Federico Ramón Gasquez\inst{3}}
%
\authorrunning{Cornejo Fandos, Di Cunzolo}   % abbreviated author list (for running head)
%

%
\institute{
13566/6, Licenciatura en Informática, Facultad de Informática, UNLP
\and
13572/5, Licenciatura en Informática, Facultad de Informática, UNLP
\and
13598/6, Licenciatura en Informática, Facultad de Informática, UNLP
}

\maketitle              % typeset the title of the contribution
% \index{Ekeland, Ivar} % entries for the author index
% \index{Temam, Roger}  % of the whole volume
% \index{Dean, Jeffrey}
%

\section{Ejercicio 1}

\textbf{Traduzca al lenguaje simbólico los siguientes enunciados:} \\

\begin{itemize}
  \item \textbf{\textit{Juan necesita un matemático o un informático.}} \\

  Sean $p$ y $q$ enunciados tales que $p$: Juan necesita un matemático, $q$: Juan necesita un informático. \\

  $p \vee q$ \\

  \item \textbf{\textit{Si Juan necesita un informático entonces necesita un matemático.}} \\

  Sean $p$ y $q$ enunciados tales que $p$: Juan necesita un matemático, $q$: Juan necesita un informático. \\

  $p \rightarrow q$ \\

  \item \textbf{\textit{Si Juan no necesita un matemático entonces necesita un informático.}} \\

  Sean $p$ y $q$ enunciados tales que $p$: Juan necesita un matemático, $q$: Juan necesita un informático. \\

  $\neg p \rightarrow q$ \\

  \item \textbf{\textit{Si Juan contrata un informático entonces el proyecto tendrá éxito.}} \\

  Sean $p$ y $q$ enunciados tales que $p$: Juan contrata un informático, $q$: El proyecto tiene exito. \\

  $p \rightarrow q$ \\

  \item \textbf{\textit{Si el proyecto no tiene éxito entonces Juan no ha contratado un informático.}} \\

  Sean $p$ y $q$ enunciados tales que $p$: El proyecto tiene éxito, $q$: Juan ha contratado un informático. \\

  $\neg p \rightarrow \neg q$ \\

  \item \textbf{\textit{El proyecto tendrá éxito si y sólo si Juan contrata un informático.}} \\

  Sean $p$ y $q$ enunciados tales que $p$: El proyecto tiene éxito, $q$: Juan contrata un informático. \\

  $p \leftrightarrow q$ \\

  \item \textbf{\textit{Para aprobar Lógica, el alumno debe asistir a clase, desarrollar un cuaderno
de prácticas aceptable y demostrar que dicho cuaderno ha sido desarrollado
por él; o desarrollar un cuaderno de prácticas aceptable y aprobar
el examen final.}} \\

  Sean $p$, $q$, $r$, $t$, $w$ tales que:

  \begin{itemize}
    \item $p$: El alumno asiste a clase
    \item $q$: El alumno desarrolla un cuaderno de prácticas aceptable
    \item $r$: El alumno demuestra que el cuaderno de prácticas ha sido desarrollado por él
    \item $t$: El alumno aprueba el examen final
    \item $w$: El alumno aprueba Lógica \\
  \end{itemize}

  $((p \wedge q \wedge r) \vee (q \wedge t)) \rightarrow w$ \\

  \item \textbf{\textit{El alumno puede asistir a clase u optar por un examen libre.}} \\

  Sean $p$ y $q$ tales que $p$: El alumno asiste a clases, $q$: El alumno opta por un examen final. \\

  $p \vee q$ \\

  \item \textbf{\textit{Si x es un número racional e y es un entero, entonces z no es real.}} \\

  Sean $p$, $q$ y $r$ tales que $p$: x es un número racional, $q$: x es un número entero, $r$: z es un número real. \\

  $(p \wedge q) \rightarrow r$ \\

  \item \textbf{\textit{La suma de dos números es par si y sólo si los dos números son pares o
los dos números son impares.}} \\

  Sean $p$, $q$, $r$ tales que $p$: La suma de dos números es par, $q$: Los dos números son pares, $r$: Los dos números son impares. \\

  $p \leftrightarrow (q \vee r)$

\end{itemize}

\section{Ejercicio 2}

\textbf{Dada la siguiente información:} \\

\textit{Si el unicornio es mítico, entonces es inmortal, pero si no es mítico, entonces
es un mamífero mortal. Si el unicornio es o inmortal o un mamífero, entonces
tiene un cuerno. El unicornio es mágico si tiene un cuerno.} \\

Sean las siguientes proposiciones:

\begin{itemize}
  \item $p$: El unicornio es mítico.
  \item $q$: El unicornio es mortal.
  \item $r$: El unicornio es un mamífero.
  \item $s$: El unicornio tiene un cuerno.
  \item $t$: El unicornio es mágico.
\end{itemize}

De la consigna se obtienen los siguientes enunciados:

\begin{itemize}
  \item $p \rightarrow \neg q$
  \item $\neg p \rightarrow (r \wedge q)$
  \item $(\neg q \vee r) \rightarrow s$
  \item $s \rightarrow t$ \\
\end{itemize}

\textbf{Simbolizarla en el Cálculo de Enunciados.} \\

\textbf{(a) El unicornio es mítico?. Fundamentar.} \\

Para determinar si el unicornio es mítico, ó no, necesitamos saber si $w \rightarrow p$ es una tautología, siendo  $w = (p \rightarrow \neg q) \wedge (\neg p \rightarrow (r \wedge q)) \wedge ((\neg q \vee r) \rightarrow s) \wedge (s \rightarrow t)$, \\

Luego, buscamos una configuración tal que $w \rightarrow p$ sea \textit{falso}, es decir, una configuración en la que, siendo $p$ falso, $w$ sea verdadero, por definición de condicional. \\

Finalmente dicha configuración existe, y es cuando $p$ es falso y el resto de las proposiciones son verdaderas. Por lo tanto $w \rightarrow p$ no es una tautología y el unicornio no es mítico. \\

\textbf{(b) El unicornio es mágico?. Fundamentar.} \\

Para determinar si el unicornio es mágico, ó no, necesitamos saber si $w \rightarrow t$ es una tautología, siendo  $w = (p \rightarrow \neg q) \wedge (\neg p \rightarrow (r \wedge q)) \wedge ((\neg q \vee r) \rightarrow s) \wedge (s \rightarrow t)$. \\

Luego, buscamos una configuración tal que $w \rightarrow t$ sea \textit{falso}, es decir, una configuración en la que, siendo $t$ falso, $w$ sea verdadero, por definición de condicional. \\

Finalmente dicha configuración no existe cuando $t$ es falso dado que buscando una configuración en la que $w$ sea verdadero nos queda siempre $w$ falso. Por lo tanto $w \rightarrow t$ es una tautología y el unicornio es mágico.

\section{Ejercicio 3}

\textbf{Se sabe que:} \\

\textit{La página web tiene un error o el examen de álgebra no es el 2 de julio.
Si el examen de álgebra es el 2 de julio entonces la página web tiene un error.
El examen de álgebra es el 14 de julio si y sólo si la página web tiene un error
y el período de exámenes no termina el 10 de julio} \\

Sean las siguientes proposiciones:

\begin{itemize}
  \item $p$: La pagina web tiene un error.
  \item $q$: El examen de álgebra es el 2 de julio.
  \item $r$: El examen de álgebra es el 14 de julio.
  \item $w$: El período de exámenes termina el 10 de julio.
\end{itemize}

De la consigna se obtienen los siguientes enunciados:

\begin{itemize}
  \item $p \vee \neg q$
  \item $q \rightarrow p$
  \item $r \leftrightarrow (p \wedge \neg w)$
\end{itemize}

\textit{Teniendo en cuenta que el período de exámenes termina el 10 de julio y que la página web tiene un error, deducir la verdad o falsedad de los siguientes enunciados:} \\

\textbf{(a) El examen de álgebra es el 2 de julio.} \\

Para determinar si el examen de álgebra es el 2 de julio, ó no, necesitamos saber si $w \rightarrow q$ es una tautología, siendo  $w = (p \vee \neg q) \wedge (q \rightarrow p) \wedge (r \leftrightarrow (p \wedge \neg w))$. \\

Luego, buscamos una configuración tal que $w \rightarrow q$ sea \textit{falso}, es decir, una configuración en la que, siendo $q$ falso, $w$ sea verdadero, por definición de condicional. \\

Finalmente dicha configuración existe, y es cuando $q, r$ son falsas y $p, w$ son verdaderas. Por lo tanto $w \rightarrow q$ no es una tautología y el examen de álgebra no es el 2 de julio. \\

\textbf{(b) Si la página web no tiene un error entonces el examen de álgebra es el 14 de julio.} \\

El consecuente queda como $\neg p \rightarrow r$. Luego el mismo nunca es falso si $p$ es verdadero. \\

\textbf{(c) $p \rightarrow r$}

no se cumple ya que si $p \rightarrow q$ es Falso, existe una configuración en la que el resto es Verdadero y es cuando $p = V, r = F, w = V$.

\section{Ejercicio 4}

\textit{Se tienen las siguientes premisas: Si Juan tiene suerte y llueve entonces estudia. Juan aprobará si y sólo si estudia o tiene suerte. Si Juan no tiene suerte entonces no llueve.} \\

Sean las siguientes proposiciones:

\begin{itemize}
  \item $p$: Juan tiene suerte
  \item $q$: Llueve
  \item $r$: Juan estudia
  \item $w$: Juan aprueba
\end{itemize}

De la consigna se obtienen los siguientes enunciados:

\begin{itemize}
  \item $(p \wedge q) \rightarrow r$
  \item $w \leftrightarrow (r \vee p)$
  \item $\neg p \rightarrow \neg q$
\end{itemize}

\textbf{Sabiendo que llueve, responder:} \\

\textbf{(a) ¿Aprobará Juan?} \\ 

Para determinar si Juan aprueba, ó no, necesitamos saber si $\lambda \rightarrow w$ es una tautología, siendo  $\lambda = ((p \wedge q) \rightarrow r) \wedge (w \leftrightarrow (r \vee p)) \wedge (\neg p \rightarrow \neg q)$. \\

Luego, buscamos una configuración tal que $\lambda \rightarrow w$ sea \textit{falso}, es decir, una configuración en la que, siendo $w$ falso, $\lambda$ sea verdadero, por definición de condicional. \\

Finalmente dicha configuración existe, y es cuando $w$ es falsa y $p, q, r$ son verdaderas. Por lo tanto $\lambda \rightarrow w$ no es una tautología y Juan no aprueba. \\

\textbf{(b) ¿Tendrá suerte Juan?} \\

Para determinar si Juan tiene suerte, ó no, necesitamos saber si $\lambda \rightarrow p$ es una tautología, siendo  $\lambda = ((p \wedge q) \rightarrow r) \wedge (w \leftrightarrow (r \vee p)) \wedge (\neg p \rightarrow \neg q)$. \\

Luego, buscamos una configuración tal que $\lambda \rightarrow p$ sea \textit{falso}, es decir, una configuración en la que, siendo $p$ falso, $\lambda$ sea verdadero, por definición de condicional. \\

Finalmente dicha configuración no existe cuando llueve, $q$ es verdadero. Por lo tanto $\lambda \rightarrow p$ es una tautología y Juan tiene suerte. \\

\end{document}
