%%%%%%%%%%%%%%%%%%%%%%%%%%%%%%%%%%%%%%%%%%%%%%%%%%%%%%%%%%%%%%%%%%%%%%%%%%%%%%%
%                         File: osa-revtex4-1.tex                             %
%                        Date: April 15, 2013                                 %
%                                                                             %
%                              BETA VERSION!                                  %
%                   JOSA A, JOSA B, Applied Optics, Optics Letters            %
%                                                                             %
%            This file requires the substyle file osajnl4-1.rtx,              %
%                   running under REVTeX 4.1 and LaTeX 2e                     %
%                                                                             %
%                   USE THE FOLLOWING REVTeX 4-1 OPTIONS:                     %
% \documentclass[osajnl,twocolumn,showpacs,superscriptaddress,10pt]{revtex4-1}%
%                    %% Use 11pt for Applied Optics                           %
%                                                                             %
%               (c) 2013 The Optical Society of America                       %
%                                                                             %
%%%%%%%%%%%%%%%%%%%%%%%%%%%%%%%%%%%%%%%%%%%%%%%%%%%%%%%%%%%%%%%%%%%%%%%%%%%%%%%

\documentclass[osajnl,twocolumn,showpacs,superscriptaddress,10pt]{revtex4-1} %% use 10pt for Applied Optics
%%\documentclass[osajnl,preprint,showpacs,superscriptaddress,12pt]{revtex4-1} %% use 12pt for preprint option
\usepackage{amsmath,amssymb,graphicx,float,enumerate}
\usepackage[cache=false]{minted}
\usepackage[utf8]{inputenc}
\graphicspath{ {images/} }

\usepackage{silence}
\WarningFilter{revtex4-1}{Repair the float}

\begin{document}

\title{Logica e Inteligencia Artificial}

\author{Ulises Jeremias Cornejo Fandos}
\affiliation{13566/7, Licenciatura en Informatica, Facultad de Informatica, UNLP}

\author{Lucas Di Cunzolo}
\affiliation{13572/5, Licenciatura en Informatica, Facultad de Informatica, UNLP}

\author{Federico Ramón Gasquez}
\affiliation{13598/6, Licenciatura en Informatica, Facultad de Informatica, UNLP}


%%\begin{abstract}
%%\end{abstract}

\maketitle %% required

\onecolumngrid

\section{Ejercicio 1}

\textit{Sean $A$, $B$ fbfs que cumplen que $(\neg A \vee B)$ es tautología. Sea C una fbf cualquiera. Determinar, si es posible, cuáles de las siguientes fbfs son tautologías y cuáles contradicciones. Justificar las respuestas.} \\

\begin{enumerate}[i-]
  \item $((\neg (A \rightarrow B)) \rightarrow C)$ \\

  Sean las variables de enunciados $p$ y $q$, sabemos que

  \begin{equation}
    (\neg p \vee q) \leftrightarrow (p \rightarrow q)
  \end{equation}

  es tautología. De la proposición \textit{1.10} se deduce que para formas enunciativas cualesquiera $A, B$,\

  \begin{equation}
    (\neg A \vee B) \leftrightarrow (A \rightarrow B)
  \end{equation}

  es tambien una tautología. Por lo tanto, $(\neg A \vee B)$ es logicamente equivalente a $(A \rightarrow B)$ y, reemplazando en el enunciado original, queda la siguiente formula:

  \begin{equation}
    ((\neg (\neg A \vee B)) \rightarrow C
    \label{eq:ejer1-1}
  \end{equation}

  Luego, sabemos por hipótesis que $(\neg A \vee B)$ es siempre verdadero para todo par de enunciados $A, B$, pues es una tautología. Entonces, dado que el antecedente del enunciado \ref{eq:ejer1-1}, sabemos que el mismo será \textit{falso} para todo par de enunciados $A, B$. Finalmente, independientemente del valor de verdad de la fbf $C$, sabemos que el enunciado completo será siempre verdadero, pues la unica configuración posible que permite que una implicación sea falsa es cuando el antecedente es verdadero y el consecuente falso. \\

  Por lo tanto, la fbf $((\neg (A \rightarrow B)) \rightarrow C)$ es una tautología. \\

  \item $(C \rightarrow ((\neg A) \vee B))$ \\

  Sean $A, B$ fbfs, sabemos que $((\neg A) \vee B)$ es una tautología por hipotesis. Luego, dado el enunciado

  \begin{equation}
    (C \rightarrow ((\neg A) \vee B))
  \end{equation}

  sabemos que el consecuente de la implicación que lo define será siempre verdadero, pues como meciona anteriormente, el mismo es una tautología. Finalmente, independientemente del valor de verdad de la fbf $C$, la implicación será siempre verdadera, pues la unica configuración posible que permite que una implicación sea falsa es cuando el antecedente es verdadero y el consecuente falso. Dado que esta configuración no tiene lugar en el enunciado, el valor de verdad es siempre verdadero. \\

  Por lo tanto, la fbf $(C \rightarrow ((\neg A) \vee B))$ es una tautología. \\

  \item $((\neg A) \rightarrow B)$ \\

  Sean $A, B$ fbfs, sabemos que $((\neg A) \vee B)$ es una tautología por hipotesis. Luego, dado el enunciado

  \begin{equation}
    ((\neg A) \rightarrow B)
    \label{eq:ejer1-3}
  \end{equation}

  comparamos el valor de verdad del enunciado $((\neg A) \rightarrow B) \leftrightarrow ((\neg A) \vee B)$ para verificar que se mantenga el valor de verdad. Si el mismo da una tautología, entonces el enunciado \ref{eq:ejer1-3} será una tautología. \\

  Sabemos que $(\neg A \vee B)$ es tautologia. Luego, independientemente de la composición interna de $A$ y $B$, sabemos que no existe configuración en la que $A$ sea verdadero y $B$ sea falso, pues sino no sería tautología, pero bien existe una configuración en la que $A$ es falso y $B$ es verdadero. \\
  
  Luego, si aplicacmos esta configuracion a $((\neg A) \rightarrow B)$, el mismo no es una tautología. \\
\end{enumerate}

\section{Ejercicio 2}

\textit{¿Es cierto que dadas $A$ y $B$ fbfs cualesquiera, siempre ocurre que si $A$ y $A \rightarrow B$ son tautologías entonces $B$ también lo es? Fundamentar. Ejemplificar con algunos ejemplos concretos escritos en lenguaje natural.} \\

La forma de argumentación planteada se compone de dos premisas. Dadas dos fbfs cualquiera, $A, B$, la primer premisa será de la forma $A \rightarrow B$ y la segunda será $A$. Luego, siendo las premisas tautologías,
evaluamos los casos en los que la implicación que define a la primer premisa sea verdadera. Es decir, siendo $A$ una tautología, nos quedamos con el caso en el que $B$ no haga del valor de la implicación una falsedad. \\

Por lo tanto, siendo $A$ una fbf siempre verdadera, $B$ debe ser verdadera para que $A \rightarrow B$ siga siendo una tautología y mantenga su valor de verdad. \\

Es por esto que, dadas las premisas $A$ y $A \rightarrow B$, si las mismas son tautologías, $B$ también lo es. \\

\textbf{\textit{Ejemplo}} \\

Sean $A$: Hoy es martes, $B$: Juan se irá a trabajar. Luego, $A \rightarrow B$: Si hoy es martes, entonces Juan se irá a trabajar. Si $A$ y $A \rightarrow B$ entonces $B$, se cumple y es válido. \\

Finalmente, este argumento es válido, pero esto no nos dice nada sobre si las premisas requeridas por el argumento son verdaderas. Para que este argumento sea un argumento sólido además de válido las premisas deberán ser verdaderas. \\

Un argumento válido pero sin solidez podría ser o no falso. El argumento de ejemplo solo es sólido los martes y cuando en efecto, se sabe que Juan realmente va a trabajar los martes. \\

\section{Ejercicio 3}

\textit{Sea $A$ una fbf donde aparecen sólo los conectivos $\wedge, \vee, \neg$ . Sea $A'$ la fbf que se obtiene a partir de $A$ reemplazando cada $\wedge$ por $\vee$ y cada $\vee$ por $\wedge$. ¿Si A es una tautología, $A'$ también lo es? Justificar. Ejemplificar con algunos ejemplos escritos en lenguaje natural.} \\

Sea $p$ una variables enunciativa cualquiera y sea $A = p \vee \neg p$, luego $A$ es tautología. \\

Luego, sea $A' = p \wedge \neg p$, por definición de $A'$, la misma no es tautología sino una contradicción. Luego, podemos ver que no para todo $A$ que sea una tautología se cumple que $A'$ tambien lo sea, por lo que es falso. \\

\section{Ejercicio 4}

\textit{Demostrar que cualquier tautología proposicional que esté escrita usando los conectivos $\neg$, $\vee$, $\wedge$, $\rightarrow$ contiene alguna ocurrencia ya sea del símbolo ”$\neg$” o del símbolo ”$\rightarrow$”.} \\

\textit{Idea: Demostrar que cualquier fórmula que contenga sólo la conjunción y disyunción puede tomar el valor $F$.} \\



\section{Ejercicio 5}

\textit{¿Es cierto que en el Cálculo de Enunciados pueden escribirse dos fbfs que tengan diferentes letras de proposición y aún así ambas fbfs sean lógicamente equivalentes?. Fundar.} \\

Al momento de evaluar la equivalencia lógica de dos fbfs, en el cálculo de enunciados, se ignora en cierta forma la composición interna de las mismas siendo que lo único relevante será siempre el valor de verdad resultante de cada una.

\newpage

\section{Ejercicio 6}

\textit{Determinar cuáles de las siguientes fbfs son lógicamente implicadas por la fbf $(A \wedge B)$. Fundamentar. Def. de implicación lógica, ver def. 1.7 del Hamilton.} \\

\begin{enumerate}[i-]
  \item $A$

  \begin{table}[h!]
    \setlength{\tabcolsep}{1.0em}
    \centering
    \begin{tabular}{ccc|c|c}
      $((A$ & $\wedge$ & $B)$ & $\rightarrow$ & $A)$ \\
      \hline
      $V$ & $V$ & $V$ & $V$ & $V$ \\
      $V$ & $F$ & $F$ & $V$ & $V$ \\
      $F$ & $F$ & $V$ & $V$ & $F$ \\
      $F$ & $F$ & $F$ & $V$ & $F$
    \end{tabular}
  \end{table}

  Luego, dado que la implicación es una tautología, $(A \wedge B)$ implica logicamente a $A$. \\

  \item $B$

  \begin{table}[h!]
    \setlength{\tabcolsep}{1.0em}
    \centering
    \begin{tabular}{ccc|c|c}
      $((A$ & $\wedge$ & $B)$ & $\rightarrow$ & $B)$ \\
      \hline
      $V$ & $V$ & $V$ & $V$ & $V$ \\
      $V$ & $F$ & $F$ & $V$ & $F$ \\
      $F$ & $F$ & $V$ & $V$ & $V$ \\
      $F$ & $F$ & $F$ & $V$ & $F$
    \end{tabular}
  \end{table}

  Luego, dado que la implicación es una tautología, $(A \wedge B)$ implica logicamente a $B$. \\

  \item $A \vee B$

  \begin{table}[h!]
    \setlength{\tabcolsep}{1.0em}
    \centering
    \begin{tabular}{ccc|c|ccc}
      $((A$ & $\wedge$ & $B)$ & $\rightarrow$ & $(A$ & $\vee$ & $B))$ \\
      \hline
      $V$ & $V$ & $V$ & $V$ & $V$ & $V$ & $V$ \\
      $V$ & $F$ & $F$ & $V$ & $V$ & $V$ & $F$ \\
      $F$ & $F$ & $V$ & $V$ & $F$ & $V$ & $V$ \\
      $F$ & $F$ & $F$ & $V$ & $F$ & $F$ & $F$
    \end{tabular}
  \end{table}

  Luego, dado que la implicación es una tautología, $(A \wedge B)$ implica logicamente a $A \vee B$. \\

  \item $\neg A \vee B$

  \begin{table}[h!]
    \setlength{\tabcolsep}{1.0em}
    \centering
    \begin{tabular}{ccc|c|cccc}
      $((A$ & $\wedge$ & $B)$ & $\rightarrow$ & $((\neg$ & $A)$ & $\vee$ & $B))$ \\
      \hline
      $V$ & $V$ & $V$ & $V$ & $F$ & $V$ & $V$ & $V$ \\
      $V$ & $F$ & $F$ & $V$ & $F$ & $V$ & $F$ & $F$ \\
      $F$ & $F$ & $V$ & $V$ & $V$ & $F$ & $V$ & $V$ \\
      $F$ & $F$ & $F$ & $V$ & $V$ & $F$ & $V$ & $F$
    \end{tabular}
  \end{table}

  Luego, dado que la implicación es una tautología, $(A \wedge B)$ implica logicamente a $\neg A \vee B$. \\

  \item $\neg B \rightarrow A$

  \begin{table}[h!]
    \setlength{\tabcolsep}{1.0em}
    \centering
    \begin{tabular}{ccc|c|cccc}
      $((A$ & $\wedge$ & $B)$ & $\rightarrow$ & $((\neg$ & $B)$ & $\rightarrow$ & $A))$ \\
      \hline
      $V$ & $V$ & $V$ & $V$ & $F$ & $V$ & $V$ & $V$ \\
      $V$ & $F$ & $F$ & $V$ & $V$ & $F$ & $F$ & $V$ \\
      $F$ & $F$ & $V$ & $V$ & $F$ & $V$ & $V$ & $F$ \\
      $F$ & $F$ & $F$ & $V$ & $V$ & $F$ & $F$ & $F$
    \end{tabular}
  \end{table}

  Luego, dado que la implicación es una tautología, $(A \wedge B)$ implica logicamente a $\neg B \rightarrow A$. \\

  \item $A \leftrightarrow B$

  \begin{table}[h!]
    \setlength{\tabcolsep}{1.0em}
    \centering
    \begin{tabular}{ccc|c|ccc}
      $((A$ & $\wedge$ & $B)$ & $\rightarrow$ & $(A$ & $\leftrightarrow$ & $B)$ \\
      \hline
      $V$ & $V$ & $V$ & $V$ & $V$ & $V$ & $V$ \\
      $V$ & $F$ & $F$ & $V$ & $V$ & $F$ & $F$ \\
      $F$ & $F$ & $V$ & $V$ & $F$ & $F$ & $V$ \\
      $F$ & $F$ & $F$ & $V$ & $F$ & $V$ & $F$
    \end{tabular}
  \end{table}

  Luego, dado que la implicación es una tautología, $(A \wedge B)$ implica logicamente a $A \leftrightarrow B$. \\

  \item $A \rightarrow B$

  \begin{table}[h!]
    \setlength{\tabcolsep}{1.0em}
    \centering
    \begin{tabular}{ccc|c|ccc}
      $((A$ & $\wedge$ & $B)$ & $\rightarrow$ & $(A$ & $\rightarrow$ & $B)$ \\
      \hline
      $V$ & $V$ & $V$ & $V$ & $V$ & $V$ & $V$ \\
      $V$ & $F$ & $F$ & $V$ & $V$ & $F$ & $F$ \\
      $F$ & $F$ & $V$ & $V$ & $F$ & $V$ & $V$ \\
      $F$ & $F$ & $F$ & $V$ & $F$ & $V$ & $F$
    \end{tabular}
  \end{table}

  Luego, dado que la implicación es una tautología, $(A \wedge B)$ implica logicamente a $A \rightarrow B$. \\

  \item $\neg B \rightarrow \neg A$

  \begin{table}[h!]
    \setlength{\tabcolsep}{1.0em}
    \centering
    \begin{tabular}{ccc|c|ccccc}
      $((A$ & $\wedge$ & $B)$ & $\rightarrow$ & $((\neg$ & $A)$ & $\rightarrow$ & $(\neg$ & $B))$ \\
      \hline
      $V$ & $V$ & $V$ & $V$ & $F$ & $V$ & $V$ & $F$ & $V$ \\
      $V$ & $F$ & $F$ & $V$ & $V$ & $F$ & $F$ & $F$ & $V$ \\
      $F$ & $F$ & $V$ & $V$ & $F$ & $V$ & $V$ & $V$ & $F$ \\
      $F$ & $F$ & $F$ & $V$ & $V$ & $F$ & $V$ & $V$ & $F$
    \end{tabular}
  \end{table}

  Luego, dado que la implicación es una tautología, $(A \wedge B)$ implica logicamente a $\neg B \rightarrow \neg A$. \\

  \item $B \rightarrow \neg A$

  \begin{table}[h!]
    \setlength{\tabcolsep}{1.0em}
    \centering
    \begin{tabular}{ccc|c|cccc}
      $((A$ & $\wedge$ & $B)$ & $\rightarrow$ & $(B$ & $\rightarrow$ & $(\neg$ & $A)))$ \\
      \hline
      $V$ & $V$ & $V$ & $F$ & $V$ & $F$ & $F$ & $V$ \\
      $V$ & $F$ & $F$ & $V$ & $F$ & $V$ & $F$ & $V$ \\
      $F$ & $F$ & $V$ & $V$ & $V$ & $V$ & $V$ & $F$ \\
      $F$ & $F$ & $F$ & $V$ & $F$ & $V$ & $V$ & $F$
    \end{tabular}
  \end{table}

  Luego, dado que la implicación no es una tautología, $(A \wedge B)$ no implica logicamente a $B \rightarrow \neg A$. \\
\end{enumerate}

\section{Ejercicio 7}

Sea la relación $\leq$ tal que dadas fbfs $A$, $B$ se cumple que $A \leq B$ sii $A \rightarrow B$ es una tautología. Dadas las fbfs: $p$, $p \rightarrow q$, $\neg p$, $p \wedge \neg p$, $r \vee \neg r$, organizarlas bajo la relación $\leq$. Representar gráficamente. \\

Primero sabemos que $p \vee \neg p$ es falso para todo enunciado $p$ y que $r \vee \neg r$ es tautología. Luego, sabiendo esto se organizan las fbfs de la siguiente forma: \\

\begin{equation}
  p \wedge \neg p; \neg p; p \rightarrow q; p; r \vee \neg r
  \label{eq:ejer7}
\end{equation}

Vale destacar que ninguna de las proposiciones, exceptuando la primera en orden, implica a $p$. Además, sabemos que $p$ implica únicamente a $r \vee \neg r$. Es por eso que el orden correcto quitando $p$ sería $p \wedge \neg p; \neg p; p \rightarrow q; r \vee \neg r$. Por lo tando en el orden mostrado en (\ref{eq:ejer7}) se opta por asociar a $p$ con la única proposición a la cual implica. \\

\section{Ejercicio 8}

\textit{Sea A una fbf donde aparecen sólo los conectivos $\wedge$, $\neg$. Sea $A'$ la fbf que se obtiene a partir de A reemplazando cada $\wedge$ por $\vee$ y cada letra de proposición por su negación (o sea, cada $p$ por $\neg p$, cada $q$ por $\neg q$, etc.). ¿Es cierto que $A'$ es lógicamente equivalente a $\neg A$? Fundamentar. Ejemplificar con algunos ejemplos concretos escritos en lenguaje natural.} \\

Sean $p_1, p_2, \ldots, p_n$ variables de enunciado, y sean $A = ((\neg p_1) \wedge (\neg p_2) \wedge \ldots \wedge (\neg p_n))$ y $A' = ((p_1) \vee (p_2) \vee \ldots \vee (p_n))$. Se cumple que

\begin{equation}
  \left(\bigwedge_{i=1}^{n}{(\neg p_i)}\right)
\end{equation}

es logicamente equivalente a

\begin{equation}
  \left(\neg \left(\bigvee_{i=1}^{n}{p_i}\right)\right)
\end{equation}

, pues el mismo es un caso especial de la proposición 1.15 y se plantea en el corolario 1.16. El mismo se cumple siempre independientemente del valor de verdad de las variables de enunciado. \\

\newpage

\section{Ejercicio 9}

Sea \# el operador binario definido como $p\#q {=}_{def} (p \wedge \neg q) \vee (\neg p \wedge q)$. Def. de implicación lógica, ver def. 1.7 del Hamilton. \\

\textbf{\textit{Probar que \# es asociativo, es decir, $x\#(y\#z)$ es lógicamente equivalente a $(x\#y)\#z$.}}

Primero evaluamos como queda el valor de verdad de $y\#z$.

\begin{table}[h!]
  \setlength{\tabcolsep}{1.0em}
  \centering
  \begin{tabular}{cccc|c|ccccc}
    $(y$ & $\wedge$ & $(\neg$ & $z))$ & $\vee$ & $((\neg$ & $y)$ & $\wedge$ & $z)$\\
    \hline
    V & F & F & V & F & F & V & F & V\\
    V & F & F & V & F & F & V & F & V\\
    V & V & V & F & V & F & V & F & F\\
    V & V & V & F & V & F & V & F & F\\
    F & F & F & V & V & V & F & V & V\\
    F & F & F & V & V & V & F & V & V\\
    F & F & V & F & F & V & F & F & F\\
    F & F & V & F & F & V & F & F & F
  \end{tabular}
\end{table}

Luego, hacemos lo mismo para $x\#(y\#z)$.

\begin{table}[h!]
  \setlength{\tabcolsep}{1.0em}
  \centering
  \begin{tabular}{cccc|c|ccccc}
    $(x$ & $\wedge$ & $(\neg$ & $(y\#z)))$ & $\vee$ & $((\neg$ & $x)$ & $\wedge$ & $(y\#z))$\\
    \hline
    V & V & V & F & V & F & V & F & F\\
    F & F & V & F & F & V & F & F & F\\
    V & F & F & V & F & F & V & F & V\\
    F & F & F & V & V & V & F & V & V\\
    V & F & F & V & F & F & V & F & V\\
    F & F & F & V & V & V & F & V & V\\
    V & V & V & F & V & F & V & F & F\\
    F & F & V & F & V & V & F & F & F
  \end{tabular}
\end{table}

Por otro lado, evaluamos el valor de verdad de $x\#y$.

\begin{table}[h!]
  \setlength{\tabcolsep}{1.0em}
  \centering
  \begin{tabular}{cccc|c|ccccc}
    $(x$ & $\wedge$ & $(\neg$ & $y))$ & $\vee$ & $((\neg$ & $x$ & $\wedge$ & $y)$\\
    \hline
    V & F & F & V & F & F & V & F & V\\
    F & F & F & V & V & V & F & V & V\\
    V & F & F & V & F & F & V & F & V\\
    F & F & F & V & V & V & F & V & V\\
    V & V & V & F & V & F & V & F & F\\
    F & F & V & F & F & V & F & F & F\\
    V & V & V & F & V & F & V & F & F\\
    F & F & V & F & F & V & F & F & F
  \end{tabular}
\end{table}

\newpage

Luego, hacemos lo mismo para $(x\#y)\#z$.

\begin{table}[h!]
  \setlength{\tabcolsep}{1.0em}
  \centering
  \begin{tabular}{cccc|c|ccccc}
    $((x\#y)$ & $\wedge$ & $(\neg$ & $z))$ & $\vee$ & $((\neg$ & $(x\#y)$ & $\wedge$ & $z)$\\
    \hline
    F & F & F & V & V & V & F & V & V\\
    V & F & F & V & F & F & V & F & V\\
    F & F & V & F & V & V & F & V & F\\
    V & V & V & F & V & F & V & V & F\\
    V & F & F & V & V & F & V & V & V\\
    F & F & F & V & F & V & F & F & V\\
    V & V & V & F & V & F & V & V & F\\
    F & F & V & F & V & V & F & V & F
  \end{tabular}
\end{table}

Finalmente, se evalua el valor de verdad de $(x\#(y\#z)) \leftrightarrow ((x\#y)\#z)$.

\begin{table}[h!]
  \setlength{\tabcolsep}{1.0em}
  \centering
  \begin{tabular}{c|c|c}
    $(x\#(y\#z))$ & $\leftrightarrow$ & $(x\#y)\#z$\\
    \hline
    V & V & V\\
    F & V & F\\
    F & F & V\\
    V & V & V\\
    F & F & V\\
    V & F & F\\
    V & V & V\\
    V & V & V
  \end{tabular}
\end{table}

\textbf{\textit{Probar que \# es conmutativo, es decir, $y\#z$ es lógicamente equivalente a $z\#y$.}}

\begin{table}[h!]
  \setlength{\tabcolsep}{0.5em}
  \centering
  \begin{tabular}{ccccccccc|c|ccccccccc}
    $((y$ & $\wedge$ & $\neg$ & $z)$ & $\vee$ & $((\neg$ & $y)$ & $\wedge$ & $z))$ & $\leftrightarrow$ & $((z$ & $\wedge$ & $\neg$ & $y)$ & $\vee$ & $((\neg$ & $z)$ & $\wedge$ & $y))$\\
    \hline
    V & F & F & V & F & F & V & F & V & V & V & F & F & V & F & F & V & F & V\\
    V & V & V & F & V & F & V & F & F & V & F & F & F & V & V & V & F & V & V\\
    F & F & F & V & V & V & F & V & V & V & V & V & V & F & V & F & V & F & F\\
    F & F & V & F & F & V & F & F & F & V & F & F & V & F & F & V & F & F & F
  \end{tabular}
\end{table}

\textbf{\textit{$\therefore$ \# es un operador lógico que satisface la propiedad conmutativa pero no la asociativa.}} \\

\section{Ejercicio 10}

Demostrar que las siguientes fórmulas son lógicamente equivalentes. \\

\begin{enumerate}[i-]
  \item $(p \rightarrow q)$ es lógicamente equivalente a $(\neg p \vee q)$ \\
  
  \begin{table}[h!]
    \setlength{\tabcolsep}{1.0em}
    \centering
    \begin{tabular}{ccc|c|cccc}
      $((p$ & $\rightarrow$ & $q)$ & $\leftrightarrow$ & $((\neg$ & $p)$ & $\vee$ & $q))$ \\
      \hline
      V & V & V & V & F & V & V & V \\
      V & F & F & F & V & V & V & F \\
      F & V & V & V & F & F & V & V \\
      F & V & F & V & V & F & V & F
    \end{tabular}
  \end{table}
  
  \item $(p \leftrightarrow q)$ es lógicamente equivalente a $((p \rightarrow q) \wedge (q \rightarrow p))$ \\
  
  \begin{table}[h!]
    \setlength{\tabcolsep}{1.0em}
    \centering
    \begin{tabular}{ccc|c|ccccccc}
      $((p$ & $\leftrightarrow$ & $q)$ & $\leftrightarrow$ & $((p$ & $\rightarrow$ & $q)$ & $\wedge$ & $(p$ & $\rightarrow$ & $q)))$ \\
      \hline
      V & V & V & V & V & V & V & V & V & V & V \\
      V & F & F & V & V & F & F & F & F & V & V \\
      F & F & V & V & F & V & V & F & V & F & F \\
      F & V & F & V & F & V & F & V & F & V & F
    \end{tabular}
  \end{table}
  
  Luego, dado que es tautologia, son logicamente equivalentes. \\
  
  \item $\neg (p \wedge q)$ es logicamente equivalente a $(\neg p \vee \neg q)$ \\
  
  \begin{table}[h!]
    \setlength{\tabcolsep}{1.0em}
    \centering
    \begin{tabular}{cccc|c|ccccc}
      $(((\neg$ & $p)$ & $\wedge$ & $q)$ & $\leftrightarrow$ & $((\neg$ & $p)$ & $\vee$ & $(\neg$ & $q)))$ \\
      \hline
      F & V & V & V & V & F & V & F & F & V \\
      V & V & F & F & V & F & V & V & V & F \\
      V & F & F & V & V & V & F & V & F & V \\
      V & F & F & F & V & V & F & V & V & F
    \end{tabular}
  \end{table}
  
  Luego, dado que es tautologia, son logicamente equivalentes. \\
  
  \item $\neg (p \vee q)$ es logicamente equivalente a $(\neg p \wedge \neg q)$ \\
  
  \begin{table}[h!]
    \setlength{\tabcolsep}{1.0em}
    \centering
    \begin{tabular}{cccc|c|ccccc}
      $(((\neg$ & $p)$ & $\vee$ & $q)$ & $\leftrightarrow$ & $((\neg$ & $p)$ & $\wedge$ & $(\neg$ & $q)))$ \\
      \hline
      F & V & V & V & V & F & V & F & F & V \\
      V & V & V & F & V & F & V & F & V & F \\
      V & F & V & V & V & V & F & F & F & V \\
      V & F & F & F & V & V & F & V & V & F
    \end{tabular}
  \end{table}
  
  Luego, dado que es tautologia, son logicamente equivalentes. \\
\end{enumerate}

\end{document}
