%%%%%%%%%%%%%%%%%%%%%%%%%%%%%%%%%%%%%%%%%%%%%%%%%%%%%%%%%%%%%%%%%%%%%%%%%%%%%%%
%                         File: osa-revtex4-1.tex                             %
%                        Date: April 15, 2013                                 %
%                                                                             %
%                              BETA VERSION!                                  %
%                   JOSA A, JOSA B, Applied Optics, Optics Letters            %
%                                                                             %
%            This file requires the substyle file osajnl4-1.rtx,              %
%                   running under REVTeX 4.1 and LaTeX 2e                     %
%                                                                             %
%                   USE THE FOLLOWING REVTeX 4-1 OPTIONS:                     %
% \documentclass[osajnl,twocolumn,showpacs,superscriptaddress,10pt]{revtex4-1}%
%                    %% Use 11pt for Applied Optics                           %
%                                                                             %
%               (c) 2013 The Optical Society of America                       %
%                                                                             %
%%%%%%%%%%%%%%%%%%%%%%%%%%%%%%%%%%%%%%%%%%%%%%%%%%%%%%%%%%%%%%%%%%%%%%%%%%%%%%%

\documentclass[osajnl,twocolumn,showpacs,superscriptaddress,10pt,leqno]{revtex4-1} %% use 10pt for Applied Optics
%%\documentclass[osajnl,preprint,showpacs,superscriptaddress,12pt]{revtex4-1} %% use 12pt for preprint option
\usepackage{amsmath,amssymb,graphicx,float,enumerate}
\usepackage[cache=false]{minted}
\usepackage[utf8]{inputenc}
\graphicspath{ {images/} }

\usepackage{silence}
\WarningFilter{revtex4-1}{Repair the float}

\usepackage{amsmath}
\usepackage{etoolbox}
\AtBeginEnvironment{align}{\setcounter{equation}{0}}

\begin{document}

\title{Logica e Inteligencia Artificial}

\author{Ulises Jeremias Cornejo Fandos}
\affiliation{13566/7, Licenciatura en Informatica, Facultad de Informatica, UNLP}

\author{Lucas Di Cunzolo}
\affiliation{13572/5, Licenciatura en Informatica, Facultad de Informatica, UNLP}

\author{Federico Ramón Gasquez}
\affiliation{13598/6, Licenciatura en Informatica, Facultad de Informatica, UNLP}


%%\begin{abstract}
%%\end{abstract}

\maketitle %% required

\onecolumngrid

\section{Ejercicio 1}

\textit{Sean A , B y C tres fórmulas bien formadas (fbfs) del sistema formal L. Dar una demostración sintáctica en L de los siguientes teoremas. Justificar cada paso en la derivación, indicando cuales son los axiomas instanciados y las reglas de inferencia utilizadas.}

\begin{enumerate}[i.]
    \item $\vdash_L ((\neg A \rightarrow A) \rightarrow A)$

    \begin{align}
        (\neg A \rightarrow A) && hipótesis \\
        (\neg A \rightarrow (\neg \neg (\neg A \rightarrow A) \rightarrow \neg A)) && (L_{1}) \\
        (\neg \neg (\neg A \rightarrow A) \rightarrow \neg A) \rightarrow (A \rightarrow \neg (\neg A \rightarrow A)) && (L_{3}) \\
        (\neg A \rightarrow (A \rightarrow \neg (\neg A \rightarrow A))) && (2), (3), SH \\
        (\neg A \rightarrow (A \rightarrow \neg (\neg A \rightarrow A))) \rightarrow ((\neg A \rightarrow A) \rightarrow (\neg A \rightarrow \neg (\neg A \rightarrow A))) && (L_{2}) \\
        (\neg A \rightarrow A) \rightarrow (\neg A \rightarrow \neg (\neg A \rightarrow A)) && (4), (5) MP \\
        (\neg A \rightarrow \neg (\neg A \rightarrow A)) && (1), (6) MP \\
        (\neg A \rightarrow \neg (\neg A \rightarrow A)) \rightarrow ((\neg A \rightarrow) \rightarrow A) && (L_{3}) \\
        (\neg A \rightarrow A) \rightarrow A && (7), (8) MP \\
        A && (1), (9) MP
    \end{align}

    Así pues, $(\neg A \rightarrow A) \vdash_L A$.
    
    $\therefore$ $\vdash_L ((\neg A \rightarrow A) \rightarrow A)$, por el Teorema de Deducción.

    \item $\vdash_L (\neg \neg B \rightarrow B)$

    Para este caso se escribe a continuación la demostración en $L$.

    \begin{align}
        ((B \rightarrow ((B \rightarrow B) \rightarrow B)) \rightarrow (B \rightarrow (B \rightarrow B)) \rightarrow (B \rightarrow B)) && (L_{2}) \\
        (B \rightarrow ((B \rightarrow B) \rightarrow B)) && (L_{1}) \\
        ((B \rightarrow (B \rightarrow B)) \rightarrow (B \rightarrow B)) && (1), (2) MP \\
        (B \rightarrow (B \rightarrow B)) && (L_{1}) \\
        (B \rightarrow B) && (3), (4) MP \\
        (\neg \neg B \rightarrow B)
    \end{align}    

    \item $\vdash_L ((A \rightarrow B) \rightarrow (\neg B \rightarrow \neg A))$

    El enunciado tiene la forma sintáctica del axioma $L_{3}$.
\end{enumerate}

\section{Ejercicio 2}

\textit{Sean A , B y C tres fórmulas bien formadas (fbfs) del sistema formal L. Dar una demostración sintáctica en L para la siguiente deducción. Justificar cada paso en la derivación, indicando cuales son los axiomas instanciados y las reglas de inferencia utilizadas.}

\begin{enumerate}[i.]
    \item $\{((A \rightarrow B) \rightarrow C), B\} \vdash_L (A \rightarrow C)$ \\

    \begin{align}
        ((A \rightarrow B) \rightarrow C) && hipótesis \\
        B && hipótesis \\
        A && hipótesis \\
        ...
    \end{align}
\end{enumerate}

\end{document}
