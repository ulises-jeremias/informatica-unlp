\documentclass[a4paper,10pt]{article}


\RequirePackage{color,graphicx}
\usepackage[usenames,dvipsnames]{xcolor}
\usepackage[big]{layaureo} 				%better formatting of the A4 page
% an alternative to Layaureo can be ** \usepackage{fullpage} **
\usepackage{supertabular} 				%for Grades
\usepackage{titlesec}					%custom \section
%Setup hyperref package, and colours for links
\usepackage{hyperref}
\definecolor{linkcolour}{rgb}{0,0.2,0.6}
\hypersetup{colorlinks,breaklinks,urlcolor=linkcolour, linkcolor=linkcolour}
\usepackage[utf8]{inputenc}

%Sections inspired by: 
%http://stefano.italians.nl/archives/26
\titleformat{\section}{\Large\scshape\raggedright}{}{0em}{}[\titlerule]
\titlespacing{\section}{0pt}{3pt}{20pt}
%Tweak a bit the top margin
%\addtolength{\voffset}{-1.3cm}

%Italian hyphenation for the word: ''corporations''
\hyphenation{im-pre-se}

%-------------WATERMARK TEST---------------
\usepackage[absolute]{textpos}

\setlength{\TPHorizModule}{30mm}
\setlength{\TPVertModule}{\TPHorizModule}
\textblockorigin{2mm}{0.65\paperheight}
\setlength{\parindent}{0pt}

\usepackage{vmargin}

\setpapersize{A4}
\setmargins{3.5cm}       % margen izquierdo
{2.0cm}                        % margen superior
{14.5cm}                      % anchura del texto
{24.0cm}                    % altura del texto
{10pt}                           % altura de los encabezados
{1cm}                           % espacio entre el texto y los encabezados
{0pt}                             % altura del pie de página
{2cm}                           % espacio entre el texto y el pie de página

%%%%%%%%%%%%%%%%%%%%%%%%%%%%%%%%%%%%%%%%%%%%%%%%%%%%%%%%%%%%%%%%%%

\usepackage{listings}
\usepackage{color}

\definecolor{codegreen}{rgb}{0,0.6,0}
\definecolor{codegray}{rgb}{0.5,0.5,0.5}
\definecolor{codepurple}{rgb}{0.58,0,0.82}
\definecolor{backcolour}{rgb}{0.95,0.95,0.92}

\lstdefinestyle{mystyle}{
backgroundcolor=\color{backcolour},
commentstyle=\color{codegreen},
keywordstyle=\color{magenta},
numberstyle=\tiny\color{codegray},
stringstyle=\color{codepurple},
basicstyle=\footnotesize,
breakatwhitespace=false,
breaklines=true,
captionpos=b,
keepspaces=true,
numbers=left,
numbersep=5pt,
showspaces=false,
showstringspaces=false,
showtabs=false,
tabsize=2
}

\title{Conceptos y Paradigmas de los Lenguajes de Programación}
\author{Ulises J. Cornejo Fandos}
\date{Abril 2017}

\begin{document}

\maketitle


\section{Práctica 3}

\begin{enumerate}
\item ¿Que define la semantica?


La semántica describe el significado de los símbolos, palabras y frases de un lenguaje ya sea lenguaje natural o lenguaje informático. La semantica se puede catalogar de dos formas:

\begin{itemize}
\item Semantica Estatica

Es la semantica que puede ser analizada al momento de compilación, esta corrobora que no haya variables sin declarar, verifica los tipos y atributos de las mismas, etc...
\item Semantica Dinamica

Este tipo de semantica se analiza en ejecución, debido a que en determinados lenguajes no es posible realizar siempre un analisis total mediante la semantica estatica.
\end{itemize}

\item
\begin{enumerate}
\item ¿Qué significa compilar un programa?

El proceso de compilación es aquel que traduce un programa escrito en un lenguaje de alto nivel a codigo maquina, indicando errores sintacticos, semanticos y optimizando parte del programa.

\item Describa brevemente cada uno de los pasos necesarios para compilar un programa.

\begin{itemize}
\item Pre Procesado

Este proceso solo se realiza en algunos lenguajes como C, lo que hace es reemplazar todas las Macros o constantes definidas en todo el codigo. También es el proceso encargado de incluir las librerias utilizadas.

\item Compilado

Encargado de traducir el codigo de alto nivel en un lenguaje objeto (lenguaje de maquina, assembler o de cercano bajo nivel).

\item Ensamblado (Assembler)

Encargado de traducir el codigo compilado en codigo maquina (ceros y unos).

\item Linkeado (Link-editor)

Encargado de linkear librerias de sistema y reubicar el codigo de entrada con todos sus modulos.

\item Loader

Encargado de cargar el programa a memoria, esto hace al mismo ejecutable, en codigo maquina.

\end{itemize}

\item ¿En qué paso interviene la semántica y cual es su importancia dentro de la compilación?

La semantica es analizada en la etapa de compilado, esta es indispensable a la hora de prevenir futuros errores tales como variables no declaradas, errores de tipos y en 'casteos', etc...

\end{enumerate}

\item Con respecto al punto anterior ¿es lo mismo compilar un programa que interpretarlo? Justifique su respuesta mostrando las diferencias básicas, ventajas y desventajas de cada uno.

No, si bien tanto la compilación como la interpretación de un codigo de alto nivel tienen pasos similares, estos no son los mismos, sus principales diferencias son:

\begin{itemize}
\item Tiempo de ejecución

En la interpretación, por cada sentencia se realiza el proceso de decodificación para determinar las operaciones a ejecutar y sus operandos, si la sentencia está en un proceso iterativo, se realizará la tarea tantas veces como sea requerido,la velocidad de proceso se puede ver afectada durante la ejecución. Mientras que en la compilación no se repiten lazos, el codigo de alto nivel se decodifica una unica vez.

\item Eficiencia

Al tener que decodificar linea por linea, le interpretación es mas lenta a la hora de ejecutarse, mientras que, desde el punto de vista del hardware, la compilación es mas eficiente.

\item Espacio ocupado

Como cada sentencia al ser compilada puede abarcar miles de sentencias de lenguaje maquina, el codigo compilable ocupa mas espacio, mientras que el interprete, al mantener sus sentencias en su forma original, optimiza este campo.

\item Detección de errores

Al momento de la detección de errores la interpretación puede ser mas precisa dependiendo determinados casos, el unico inconveniente es que al ser codigo dinamico, el cual no siempre sigue el mismo 'camino', puede que no nos enteremos de que parte del codigo tiene errores.

\end{itemize}

\item Explique claramente la diferencia entre un error sintáctico y uno semántico. Ejemplifique cada caso.

\begin{itemize}
\item Un error sintactico hace referencia a los errores que ocurren cuando una sentencia no esta escrita en un formato aceptable, este formato suele estar definido en un documento BNF/EBNF e indica como se definen dichas sentencias, ejemplos de esto pueden ser en un lenguaje como C:

\begin{center}

x int = 2;
\newline (en C seria int x = 2;)
\newline
\newline x++
\newline (le falta el ';')

\end{center}

\item Un error semantico hace referencia a errores de tipos, variables indeclaradas, etc... Casos comunes suelen ser (nuevamente en C):

\begin{center}

int x = 2;
\newline x = 3.5;
\newline (error de tipo)

\end{center}
\end{itemize}

\item Sean los siguientes ejemplos de programas. Analice y diga qué tipo de error se produce (Semántico o Sintáctico) y en qué momento se detectan dichos errores (Compilación o Ejecución). Aclaración: Los valores de la ayuda pueden ser mayores.

\subitem Pascal
\begin{lstlisting}[language=Pascal]
Program P
var 5: integer;
var a:char;
Begin
for i:=5 to 10 do begin
write(a);
a=a+1;
end;
End.
// Ayuda: sintacticos 2, semanticos 3
\end{lstlisting}

\subitem Java
\begin{lstlisting}[language=Java]
public String tabla(int numero, arrayList<Boolean> listado)
{
String result = null;
for(i = 1; i < 11; i--) {
result += numero + "x" + i + "=" + (i*numero) + "\n";
listado.get(listado.size()-1)=(BOOLEAN)numero>i;
}
return true;
}
Ayuda:
// Sintacticos 4, Semanticos 3, Logicos 1
\end{lstlisting}

\newpage
\subitem C
\begin{lstlisting}[language=C]
# include <stdio.h>
int suma; /* Esta es una variable global */
int main()
{ int indice;
encabezado;
for (indice = 1 ; indice <= 7 ; indice ++)
cuadrado (indice);
final(); Llama a la funcion final */
return 0;
}
cuadrado (numero)
int numero;
{ int numero_cuadrado;
numero_cuadrado == numero * numero;
suma += numero_cuadrado;
printf("El cuadrado de %d es %d\n",
numero, numero_cuadrado);
}

// Ayuda: Sintacticos 2, Semanticos 6
\end{lstlisting}

\subitem Python
\begin{lstlisting}[language=Python]
#!/usr/bin/python
print "\nDEFINICION DE NUMEROS PRIMOS"
r = 1
while r = True:
N = input("\nDame el numero a analizar: ")
i = 3
fact = 0
if (N mod 2 == 0) and (N != 2):
print "\nEl numero %d NO es primo\n" % N
else:
while i <= (N^0.5):
if (N % i) == 0:
mensaje="\nEl numero ingresado NO es primo\n" % N
msg = mensaje[4:6]
print msg
fact = 1
i+=2
if fact == 0:
print "\nEl numero %d SI es primo\n" % N
r = input("Consultar otro numero? SI (1) o NO (0)--->> ")

// Ayuda: Sintacticos 2, Semanticos 3
\end{lstlisting}

\subitem Ruby
\begin{lstlisting}[language=Ruby]
def ej1
Puts 'Hola, Cual es tu nombre?'
nom = gets.chomp
puts 'Mi nombre es ', + nom
puts 'Mi sobrenombre es 'Juan''
puts 'Tengo 10 agnos'
meses = edad*12
dias = 'meses' *30
hs= 'dias * 24'
puts 'Eso es: meses + ' meses o ' + dias + ' dias o ' + hs + ' horas
puts 'vos cuantos agnos tenes'
edad2 = gets.chomp
edad = edad + edad2.to_i
puts 'entre ambos tenemos ' + edad + ' agnos'
puts 'Sabes que hay ' + name.length.to_s + ' caracteres en tu nombre, ' + name + '?'
end
\end{lstlisting}

\item Dado el siguiente código escrito en pascal. Transcriba la misma funcionalidad de acuerdo al lenguaje que haya cursado en años anteriores. Defina brevemente la sintaxis (sin hacer la gramática) y semántica para la utilización de arreglos y estructuras de control del ejemplo.

\begin{tabular}{p{6cm}|p{7cm}}
Procedure ordenarArreglo(var arreglo: arregloDeCaracteres; cont:integer); & \\
var & \\
i:integer; ordenado:boolean; & \\
aux:char; & \\
begin & \\
repeat & \\
ordenado:=true; & \\
for i:=1 to cont-1 do & \\
if ord(arreglo[i]) $>$ ord(arreglo[i+1]) & \\
then begin & \\
aux := arreglo[i]; & \\
arreglo[i] := arreglo[i+1]; & \\
arreglo[i+1] := aux; ordenado := false & \\
end; & \\
until ordenado; & \\
end; & \\

\end{tabular}
\hspace{1cm}
Observación: Aquí sólo se debe definir la instrucción y qué es lo que hace cada una; detallando alguna particularidad del lenguaje respecto de ella. Por ejemplo el for de java necesita definir una variable entera, una condición y un incremento para dicha variable.

\newpage

\item Explique cuál es la semántica para las variables predefinidas en lenguaje Ruby 'self' y 'nil'. ¿Que valor toman? ¿Cómo son usadas por el lenguaje?

En Ruby 'self' hace referencia al objeto que se esta ejecutando en el momento, la referencia 'self' puede ser utilizada en cualquier momento del programa y tendra el contexto semantico que tiene el objeto al que referencie en ese momento de ejecución. Por otro lado, 'nil', hace referencia al valor null de otros lenguajes de programación, suele utilizarse para verificar si hay variables que no hayan sido inicializadas, u otras funciones de control, su contexto semantico siempre debe ser utilizado en relación a una variable o expresión lógica.

\item Determine la semántica de la sentencia 'break' en C, PHP, javascript y Ruby. Cíte las características más importantes de esta sentencia para cada lenguaje.

\begin{itemize}
\item 'break' en C

Se utiliza para la 'ruptura' de bucles tales como la sentencia 'for', 'while' y también para realizar una salida de estructuras como 'swich', la sentencia 'break' debe estar anidado en alguna de dichas estructuras, este es su contexto semantico.

\item 'break' en PHP

Dicha sentencia tiene el mismo contexto semantico que en C (debido a que este lenguaje deriba del mismo).

\item 'break' en javascript

Sigue la semantica de C.

\item 'break' en Ruby

Ruby utiliza 'break' al igual que C y los demás lenguajes mencionados, su diferencia es que funciones como 'switch' no existen (existen equivalencias como 'case'), mas allá de los nombres de las mismas, el contexto semantico es el mismo.

\end{itemize}

\item Defina el concepto de ligadura y su importancia respecto de la semántica de un programa. ¿Qué diferencias hay entre ligadura estática y dinámica? Cite ejemplos (proponer casos sencillos).

La ligadura hace referencia a la especificación exacta de la naturaleza de un atributo, ya sea su tipo o clase perteneciente. Esta se puede definir como:

\begin{itemize}
\item Ligadura estatica

Este tipo de ligadura se establece antes de la ejecución del programa y no es modificable durante el tiempo de vida del mismo.

\begin{center}

int x; // Liga el tipo int al atributo tipo de la variable x.

\end{center}

\item Ligadura dinamica

Se establece en el momento de la ejecución y puede cambiarse durante la misma, siguiendo las reglas que especifique cada lenguaje de programación.

\begin{center}

x = 2 // Liga el tipo int al atributo tipo de la variable x.\newline
x = 'c' // Cambia el atributo tipo de la variable x por char.

\end{center}


\end{itemize}

\end{enumerate}

\end{document}
